%!TEX TS-program = xelatex
\documentclass[10pt,oneside]{article}

\usepackage[english]{babel}

\usepackage{amsmath,amssymb,amsfonts}
\usepackage[utf8]{inputenc}
\usepackage[T1]{fontenc}
\usepackage{stix}
\usepackage[scaled]{helvet}
\usepackage[scaled]{inconsolata}

\usepackage{lastpage}

\usepackage{setspace}

\usepackage{ccicons}

\usepackage[hang,flushmargin]{footmisc}

\usepackage{geometry}

\setlength{\parindent}{0pt}
\setlength{\parskip}{6pt plus 2pt minus 1pt}

\usepackage{fancyhdr}
\renewcommand{\headrulewidth}{0pt}\providecommand{\tightlist}{%
  \setlength{\itemsep}{0pt}\setlength{\parskip}{0pt}}

\makeatletter
\newcounter{tableno}
\newenvironment{tablenos:no-prefix-table-caption}{
  \caption@ifcompatibility{}{
    \let\oldthetable\thetable
    \let\oldtheHtable\theHtable
    \renewcommand{\thetable}{tableno:\thetableno}
    \renewcommand{\theHtable}{tableno:\thetableno}
    \stepcounter{tableno}
    \captionsetup{labelformat=empty}
  }
}{
  \caption@ifcompatibility{}{
    \captionsetup{labelformat=default}
    \let\thetable\oldthetable
    \let\theHtable\oldtheHtable
    \addtocounter{table}{-1}
  }
}
\makeatother

\usepackage{array}
\newcommand{\PreserveBackslash}[1]{\let\temp=\\#1\let\\=\temp}
\let\PBS=\PreserveBackslash

\usepackage[breaklinks=true]{hyperref}
\hypersetup{colorlinks,%
citecolor=blue,%
filecolor=blue,%
linkcolor=blue,%
urlcolor=blue}
\usepackage{url}

\usepackage{caption}
\setcounter{secnumdepth}{0}
\usepackage{cleveref}

\usepackage{graphicx}
\makeatletter
\def\maxwidth{\ifdim\Gin@nat@width>\linewidth\linewidth
\else\Gin@nat@width\fi}
\makeatother
\let\Oldincludegraphics\includegraphics
\renewcommand{\includegraphics}[1]{\Oldincludegraphics[width=\maxwidth]{#1}}

\usepackage{longtable}
\usepackage{booktabs}

\usepackage{color}
\usepackage{fancyvrb}
\newcommand{\VerbBar}{|}
\newcommand{\VERB}{\Verb[commandchars=\\\{\}]}
\DefineVerbatimEnvironment{Highlighting}{Verbatim}{commandchars=\\\{\}}
% Add ',fontsize=\small' for more characters per line
\usepackage{framed}
\definecolor{shadecolor}{RGB}{248,248,248}
\newenvironment{Shaded}{\begin{snugshade}}{\end{snugshade}}
\newcommand{\KeywordTok}[1]{\textcolor[rgb]{0.13,0.29,0.53}{\textbf{#1}}}
\newcommand{\DataTypeTok}[1]{\textcolor[rgb]{0.13,0.29,0.53}{#1}}
\newcommand{\DecValTok}[1]{\textcolor[rgb]{0.00,0.00,0.81}{#1}}
\newcommand{\BaseNTok}[1]{\textcolor[rgb]{0.00,0.00,0.81}{#1}}
\newcommand{\FloatTok}[1]{\textcolor[rgb]{0.00,0.00,0.81}{#1}}
\newcommand{\ConstantTok}[1]{\textcolor[rgb]{0.00,0.00,0.00}{#1}}
\newcommand{\CharTok}[1]{\textcolor[rgb]{0.31,0.60,0.02}{#1}}
\newcommand{\SpecialCharTok}[1]{\textcolor[rgb]{0.00,0.00,0.00}{#1}}
\newcommand{\StringTok}[1]{\textcolor[rgb]{0.31,0.60,0.02}{#1}}
\newcommand{\VerbatimStringTok}[1]{\textcolor[rgb]{0.31,0.60,0.02}{#1}}
\newcommand{\SpecialStringTok}[1]{\textcolor[rgb]{0.31,0.60,0.02}{#1}}
\newcommand{\ImportTok}[1]{#1}
\newcommand{\CommentTok}[1]{\textcolor[rgb]{0.56,0.35,0.01}{\textit{#1}}}
\newcommand{\DocumentationTok}[1]{\textcolor[rgb]{0.56,0.35,0.01}{\textbf{\textit{#1}}}}
\newcommand{\AnnotationTok}[1]{\textcolor[rgb]{0.56,0.35,0.01}{\textbf{\textit{#1}}}}
\newcommand{\CommentVarTok}[1]{\textcolor[rgb]{0.56,0.35,0.01}{\textbf{\textit{#1}}}}
\newcommand{\OtherTok}[1]{\textcolor[rgb]{0.56,0.35,0.01}{#1}}
\newcommand{\FunctionTok}[1]{\textcolor[rgb]{0.00,0.00,0.00}{#1}}
\newcommand{\VariableTok}[1]{\textcolor[rgb]{0.00,0.00,0.00}{#1}}
\newcommand{\ControlFlowTok}[1]{\textcolor[rgb]{0.13,0.29,0.53}{\textbf{#1}}}
\newcommand{\OperatorTok}[1]{\textcolor[rgb]{0.81,0.36,0.00}{\textbf{#1}}}
\newcommand{\BuiltInTok}[1]{#1}
\newcommand{\ExtensionTok}[1]{#1}
\newcommand{\PreprocessorTok}[1]{\textcolor[rgb]{0.56,0.35,0.01}{\textit{#1}}}
\newcommand{\AttributeTok}[1]{\textcolor[rgb]{0.77,0.63,0.00}{#1}}
\newcommand{\RegionMarkerTok}[1]{#1}
\newcommand{\InformationTok}[1]{\textcolor[rgb]{0.56,0.35,0.01}{\textbf{\textit{#1}}}}
\newcommand{\WarningTok}[1]{\textcolor[rgb]{0.56,0.35,0.01}{\textbf{\textit{#1}}}}
\newcommand{\AlertTok}[1]{\textcolor[rgb]{0.94,0.16,0.16}{#1}}
\newcommand{\ErrorTok}[1]{\textcolor[rgb]{0.64,0.00,0.00}{\textbf{#1}}}
\newcommand{\NormalTok}[1]{#1}

\newlength{\cslhangindent}
\setlength{\cslhangindent}{1.5em}
\newlength{\csllabelwidth}
\setlength{\csllabelwidth}{3em}
\newenvironment{CSLReferences}[3] % #1 hanging-ident, #2 entry spacing
 {% don't indent paragraphs
  \setlength{\parindent}{0pt}
  % turn on hanging indent if param 1 is 1
  \ifodd #1 \everypar{\setlength{\hangindent}{\cslhangindent}}\ignorespaces\fi
  % set entry spacing
  \ifnum #2 > 0
  \setlength{\parskip}{#2\baselineskip}
  \fi
 }%
 {}
\usepackage{calc} % for \widthof, \maxof
\newcommand{\CSLBlock}[1]{#1\hfill\break}
\newcommand{\CSLLeftMargin}[1]{\parbox[t]{\maxof{\widthof{#1}}{\csllabelwidth}}{#1}}
\newcommand{\CSLRightInline}[1]{\parbox[t]{\linewidth}{#1}}
\newcommand{\CSLIndent}[1]{\hspace{\cslhangindent}#1}\usepackage[table,dvipsnames]{xcolor}

\geometry{includemp,
            letterpaper,
            top=1.2in,
            bottom=2.510cm,
            inner=0.5in,
            outer=0.4in,
            marginparwidth=1.95in,
            marginparsep=0.4in}

\usepackage[singlelinecheck=off]{caption}
\captionsetup{
  font={small},
  labelfont={bf},
  format=plain,
  labelsep=quad
}
\usepackage{floatrow}
\floatsetup[figure]{margins=hangright,
              facing=no,
              capposition=beside,
              capbesideposition={center,outside},
              floatwidth=\textwidth}
\floatsetup[table]{margins=hangoutside,
             facing=yes,
             capposition=beside,
             capbesideposition={center,outside},
             floatwidth=\textwidth}

\pagestyle{plain}

\setcounter{secnumdepth}{5}

\usepackage{titlesec}

\titleformat{\section}[block]
{\normalfont\large\sffamily}
{\thesection}{.5em}{\titlerule\\[.8ex]\bfseries}

\titleformat{\subsection}[runin]
{\normalfont\fontseries{b}\selectfont\filright\sffamily}
{\thesubsection.}{.5em}{}

\titleformat{\subsubsection}[runin]
{\normalfont\itshape\rmfamily\bfseries}{\thesubsubsection}{1em}{}

\fancypagestyle{firstpage}
{
   \fancyhf{}
   \renewcommand{\headrulewidth}{0pt}
   \fancyfoot[R]{\footnotesize\ccby}
   \fancyfoot[L]{\footnotesize\sffamily\today}
}

\fancypagestyle{normal}
{
  \fancyhf{}
  \fancyfoot[R]{\footnotesize\sffamily\thepage\ of \pageref*{LastPage}}
}

\usepackage{tikz}
\begin{document}
\tikz [remember picture, overlay] %
\node [shift={(-0.6in,1.1cm)},scale=0.2,opacity=0.4] at (current page.south east)[anchor=south east]{\includegraphics{logo}};%
\pagestyle{normal}
\thispagestyle{firstpage}

\newcommand{\colorRule}[3][black]{\textcolor[HTML]{#1}{\rule{#2}{#3}}}

\noindent {\LARGE \textbf{\textsf{Mismatch between IUCN range maps and
species interactions data illustrated using the Serengeti food web}}}

\medskip
\begin{flushleft}
{\small
%
\href{https://orcid.org/0000-0003-2791-8383}{Gracielle\,T. Higino}%
%
\,\textsuperscript{1,2,‡}, %
\href{https://orcid.org/0000-0001-5030-3470}{Fredric\,M. Windsor}%
%
\,\textsuperscript{3,‡}, %
\href{https://orcid.org/0000-0001-9051-0597}{Francis\,Banville}%
%
\,\textsuperscript{4,5,6,‡}, %
\href{https://orcid.org/0000-0002-2212-3584}{Gabriel\,Dansereau}%
%
\,\textsuperscript{4,6,‡}, %
\href{https://orcid.org/0000-0001-9019-0108}{Norma R.\,Forero-Muñoz}%
%
\,\textsuperscript{4,6,‡}, %
\href{https://orcid.org/0000-0002-0735-5184}{Timothée\,Poisot}%
%
\,\textsuperscript{4,6,‡}
\vskip 1em
\textsuperscript{1}\,The University of British
Columbia; \textsuperscript{2}\,Computational Biodiversity Science and
Services; \textsuperscript{3}\,School of Natural and Environmental
Sciences, Newcastle University; \textsuperscript{4}\,Université de
Montréal; \textsuperscript{5}\,Université de
Sherbrooke; \textsuperscript{6}\,Quebec Centre for Biodiversity
Science\\
\textsuperscript{‡}\,These authors contributed equally to the work\\
\vskip 1em
\textbf{Correspondance to:}\\
Gracielle T. Higino --- \texttt{graciellehigino@gmail.com}\\
Timothée Poisot --- \texttt{timothee.poisot@umontreal.ca}\\
}
\end{flushleft}

\vskip 2em
\makebox[0pt][l]{\colorRule[CCCCCC]{2.0\textwidth}{0.5pt}}
\vskip 2em
\noindent

\marginpar{\vskip 1em\flushright
{\small{\bfseries Keywords}:\par
range maps\\species interactions\\food web\\Serengeti\\}
}


\textbf{Abstract}:\,\textbf{Background.} Range maps are a useful tool to
describe the spatial distribution of species. However, they need to be
used with caution, as they essentially represent a rough approximation
of a species' suitable habitats. When stacked together, the resulting
communities in each grid cell may not always be realistic, especially
when species interactions are taken into account. Here we show the
extent of the mismatch between range maps, provided by the International
Union for Conservation of Nature (IUCN), and species interactions data.
More precisely, we show that local networks built from those stacked
range maps often yield unrealistic communities, where species of higher
trophic levels are completely disconnected from primary producers.
\textbf{Methodology.} We used the well-described Serengeti food web of
mammals and plants as our case study, and provide updated range maps for
all predators by taking into account food-web structure. We then used
occurrence data from the Global Biodiversity Information Facility (GBIF)
to investigate where data is most lacking. \textbf{Results.} We found
that most predator ranges comprised large areas without any overlapping
distribution of their preys. However, many of these areas contained GBIF
occurrences of the predator. \textbf{Conclusions.} Our results suggest
that the mismatch between both data sources could be due either to the
lack of information about ecological interactions or the geographical
occurrence of preys. We finally discuss general guidelines to help
identify defective data among distributions and interactions data, and
we recommend this method as a valuable way to assess whether the
occurrence data that are being used, even if incomplete, are
ecologically accurate.

\vskip 2em
\makebox[0pt][l]{\colorRule[CCCCCC]{2.0\textwidth}{0.5pt}}
\vskip 2em

\hypertarget{introduction}{%
\section{Introduction}\label{introduction}}

Finding a species in a certain location is like finding an encrypted
message that traveled through time. It carries the species' evolutionary
history, migration patterns, as well as any direct and indirect effects
generated by other species (some of which we may not even know exist).
Ecologists have been trying to decode this message with progressively
more powerful tools, from their field notes to highly complex
computational algorithms. However, to succeed in this challenge it is
important to have the right clues in hand. There are many ways we can be
misled by data - or the lack of it: taxonomic errors, geographic
inaccuracy, or sampling biases (Ladle and Hortal 2013; Hortal et al.
2015; Poisot et al. 2021). One way to identify - and potentially fix -
these errors is to combine many different pieces of information about
the occurrence of a species, so agreements and mismatches can emerge.
Here we suggest jointly analyzing species occurrence (range maps and
point occurrences) and ecological interactions to identify mismatches
between datasets.

Interactions form complex networks that shape ecological structures and
maintain the essential functions of ecosystems, such as seed dispersal,
pollination, and biological control (Albrecht 2018; Fricke et al. 2022)
that ultimately affect the composition, richness, and successional
patterns of communities across biomes. Yet, the connection between
occurrence and interaction data is a frequent debate in ecology
(Blanchet, Cazelles, and Gravel 2020). For instance, macroecological
models are often used with point or range occurrence data in order to
investigate the dynamics of a species with its environment. However,
these models do not account for ecological interactions, which might
largely affect species distribution (Abrego et al., n.d.; Afkhami,
McIntyre, and Strauss 2014; Araújo, Marcondes-Machado, and Costa 2014;
Godsoe et al. 2017; Godsoe and Harmon 2012). Some researchers argue that
occurrence data can also capture real-time interactions (Roy et al.
2016; Ryan et al. 2018), and, because of that, it would not be necessary
to include ecological interaction dynamics in macroecological models. On
the other hand, many mechanistic simulation models in ecology have
considered the effect of competition and facilitation in range shifts,
whilst the use of trophic interactions in this context remains
insufficient (Cabral, Valente, and Hartig 2017).

A significant challenge in this debate is the quality and quantity of
species distribution and ecological data (Boakes et al. 2010; Ronquillo
et al. 2020; Meyer, Weigelt, and Kreft 2016) - a gap that can lead to
erroneous conclusions in macroecological research (Hortal et al. 2008).
Amongst the geographical data available are the range maps provided by
the International Union for the Conservation of Nature (IUCN). Such maps
consist of simplified polygons, often created as alpha or convex hulls
around known species locations, refined by expert knowledge about the
species (2021). These maps can be used in macroecological inferences in
the lack of more precise information (Fourcade 2016; Alhajeri and
Fourcade 2019), but it has been recommended that they are used with
caution since they tend to underestimate the distribution of species
that are not well-known (Herkt, Skidmore, and Fahr 2017), do not
represent spatial variation in species occurrence and abundance (Dallas,
Pironon, and Santini 2020), and can include inadequate areas within the
estimated range. Another source of species distribution information is
the Global Biodiversity Information Facility (GBIF), which is an online
repository of georeferenced observational records that come from various
sources, including community science programs, museum collections, and
long-term monitoring schemes. A great source of bias in these datasets
is the irregular sampling effort, with more occurrences originating from
attractive and accessible areas and observation of charismatic species
(Alhajeri and Fourcade 2019). As for ecological data, a complete
assessment is difficult and is aggravated by biased sampling methods and
data aggregation (Poisot et al. 2020; Hortal et al. 2015). Nevertheless,
we have witnessed an increase in the availability of biodiversity data
in the last decades, including those collected through community science
projects (Callaghan et al. 2019; Pocock et al. 2015) and dedicated
databases, such as Mangal (Poisot et al. 2016). This provides an
opportunity to merge species distribution and ecological interaction
data to improve our predictions of where a species may be found across
large spatial scales (e.g., continental and global).

In this context, we elaborate a method that allows us to refine
distribution data (more precisely range maps) based on interaction data,
considering the basic assumption that predators can only be present in
regions where they are connected to at least one herbivore - and thus
indirectly connected to primary producers. We used a Serengeti food web
dataset (Baskerville et al. 2011) (which comprises carnivores,
herbivores, and plants from Tanzania) to demonstrate how a mismatch
between occurrence and interaction data can highlight significant
uncertainty areas in IUCN range maps. Finally, we add the GBIF
occurrence points for the Serengeti species to the investigation,
discuss the mechanisms that can lead to the lack of agreement between
data, and build from that a vision for the next steps, reinforcing the
importance of geographically explicit interaction data.

\hypertarget{methods}{%
\section{Methods}\label{methods}}

Organisms cannot persist unless they are directly or indirectly
connected to a primary producer within their associated food web (Power
1992). Therefore, the range of a predator (omnivore or carnivore)
depends on the overlapping ranges of its preys. If sections of a
predator's range does not overlap with at least one of its prey it will
become disconnected from primary producers, and therefore we would not
expect the predator to occur in this area. This mismatch can be the
result of different mechanisms, like the overestimation of the
predator's range, taxonomic errors, or the lack of information about
trophic links. Thus, given that herbivores are the main connection
between plant resources (directly limited by environmental conditions)
and predators (Dobson 2009; Scott et al. 2018), here we adjusted the
ranges of predators based on a simple rule: we removed any part of a
predator's range that did not intersect with the range of at least one
prey herbivore species. So, unless the range of the predator overlapped
with at least one prey item, which in turn is directly connected to a
primary producer (plants), we removed that section of the predator's
range. Finally, we calculated the difference in range size between the
original IUCN ranges and those adjusted based on species interaction
data.

\hypertarget{data}{%
\subsection{Data}\label{data}}

We investigated the mismatch between savannah species ranges and
interactions in Africa (fig.~\ref{fig:richness}). These ecosystems host
a range of different species, including the well-characterized
predator-prey dynamics between iconic predators (e.g., lions, hyenas,
and leopards) and large herbivores (e.g., antelopes, wildebeests, and
zebras), as well as a range of herbivorous and carnivorous small
mammals. The Serengeti ecosystem has been extensively studied and its
food web is one of the most complete we have to date, including primary
producers identified to the species level. Here we focus on six groups
of herbivores and carnivores from the Serengeti Food Web Data Set
(Baskerville et al. 2011). These species exhibit direct antagonistic
(predator-prey) interactions with one another and are commonly found
across savannah ecosystems on the African continent (McNaughton 1992).
Plants in the network were included indirectly in our analyses as we do
not expect the primary producers to significantly influence the range of
herbivores for several reasons. Firstly, many savannah plants are
functionally similar (i.e., grasses, trees and shrubs) and cooccur
across the same habitats (Baskerville et al. 2011). Secondly, herbivores
in the network are broadly generalists feeding on a wide range of
different plants across habitats. Indeed, out of 129 plants in our
dataset, herbivores (n = 23) had a mean out degree (mean number of
preys) of around 22 (std = 17.5). There is also an absence of global
range maps for many plant species (Daru 2020), which prevents their
direct inclusion in our analysis. Therefore, we assume that plants
consumed by herbivores are present across their ranges, and as such the
ranges of herbivores are not expected to be significantly constrained by
the availability of food plants.

From the wider ecological network presented in Baskerville (2011), we
sampled interaction data for herbivores and carnivores. This subnetwork
contained 32 taxa (23 herbivores and 9 carnivores) and 84 interactions
and had a connectance of 0.08. Although self-loops are informative, we
removed these interactions to allow for the original IUCN ranges of
predators with cannibalistic interactions to be adjusted. We treated
this overall network as a metaweb since it \emph{should} contain all
potential species interactions between mammalian taxa occurring across
savannah ecosystems such as the Serengeti.

We compiled IUCN range maps for the 32 species included in the metaweb
from the Spatial Data Download portal
(www.iucnredlist.org/resources/spatial-data-download), which we
rasterized at 10 arc-minute resolution (\textasciitilde19 km² at the
equator). We then combined interaction data from the metaweb and
cooccurrence data generated from species ranges to create networks for
each raster pixel. This generated a total of 84,244 pixel-level
networks. These networks describe potential predation, not actual
interactions: the former is derived information from the metaweb, and
the latter is contingent on the presence of herbivores.

\hypertarget{range-overlap-measurement}{%
\subsection{Range overlap measurement}\label{range-overlap-measurement}}

We calculated the geographical overlap, i.e.~the extent to which
interacting predator and prey species co-occurred across their ranges,
as \(a/(a + c)\), where \(a\) is the number of pixels where predator and
prey cooccur and \(c\) is the number of pixels where only the focal
species occur. This index of geographical overlap can be calculated with
prey or predators as the focal species. Values vary between 0 and 1,
with values closer to 1 indicating that there is a large overlap in the
ranges of the two species and values closer to 0 indicating low
cooccurrence across their ranges. For each predator species, we
calculated its generality to understand whether the level of trophic
specialization (i.e., number of prey items per predator) affects the
extent to which the ranges of the species were altered. One would assume
that predators with a greater number of prey taxa (i.e., a higher
generality) are less likely to have significant changes in their range
as it is more likely that at least one prey species is present across
most of their range.

\hypertarget{validation}{%
\subsection{Validation}\label{validation}}

For each species in the dataset we collated point observation data from
GBIF (www.gbif.org), and condensed these data into pixels representing
presence or absence of the focal taxon. These data were used to validate
the range adjustments made based on species interactions (see the
previous section). To do so, we calculated the proportion of GBIF
presence pixels occurring within both the original and adjusted species
ranges. We then compared these proportions for the predators to verify
if the range adjustments removed locations with GBIF observations, hence
likely true habitats.

\hypertarget{results}{%
\section{Results}\label{results}}

Mammal species found in the Serengeti food web are widespread in Africa,
especially in grasslands and savannahs (first panel of
fig.~\ref{fig:richness}). However, most local networks (83.2\%) built
using the original IUCN range maps had at least one mammal species
without a path to a primary producer (second panel of
fig.~\ref{fig:richness}). On average, local food webs had almost the
third of their mammal species (mean = 30.5\%, median = 14.3\%)
disconnected from basal species. In addition, many networks (16.6\%)
only had disconnected mammals; these networks however all had a very low
number of mammal species, specifically between 1 and 4 (from a total of
32). As expected, the proportion of carnivores with a path to a primary
producer was conditional on the total number of mammal species in each
local network (third panel of fig.~\ref{fig:richness}).

\begin{figure}
\hypertarget{fig:richness}{%
\centering
\includegraphics{figures/richness_prop_removed.png}
\caption{(a) Spatial distribution of species richness according to the
original IUCN range maps of all 32 mammal species of the Serengeti food
web. (b) Proportion of mammal species remaining in each local network
(i.e., each pixel) after removing all species without a path to a
primary producer. (c) Proportion of mammal species remaining in each
local network as a function of the number of species given by the
original IUCN range maps.}\label{fig:richness}
}
\end{figure}

\hypertarget{specialized-predators-lose-more-range}{%
\subsection{Specialized predators lose more
range}\label{specialized-predators-lose-more-range}}

\begin{figure}
\hypertarget{fig:degree}{%
\centering
\includegraphics{figures/rel_loss-outdegree-species.png}
\caption{Negative relationship between the out degree of predator
species and their relative range loss. More specialized predators lose a
higher proportion of their ranges due to mismatches with the ranges of
their preys.}\label{fig:degree}
}
\end{figure}

Predators with fewer prey lose more range with our method
(fig.~\ref{fig:degree}). For instance, both \emph{Leptailurus serval}
and \emph{Canis mesomelas} have only one prey in the Serengeti food web
(tbl.~\ref{tbl:everyone}), each of them with a very small range compared
to those of their predator. This discrepancy between range sizes
promotes significant range loss. On the other hand, predators of the
genus \emph{Panthera} are some of the most connected species, and they
also lose the least proportion of their ranges. This mismatch between
predators and preys can also be a result of taxonomic disagreement
between the geographical and ecological data. Although \emph{Canis
aureus} has the same number of prey as \emph{Caracal caracal}, none of
the prey taxa of the former occurs inside its original range
(tbl.~\ref{tbl:everyone}), which results in complete range loss.

\begin{figure}
\hypertarget{fig:geo_diss}{%
\centering
\includegraphics{figures/beta-div_pred-species.png}
\caption{Geographical similarity between the original IUCN range maps of
predators and preys. Dots represent predator-prey pairs, with different
symbols corresponding to different predators. For a given pair of
species, the number \(c\) of pixels where the focal species is present
but not the other and the number \(a\) of pixels where the predator and
prey cooccur, were calculated. Geographic similarities were given by
\(a/(a+c)\), with the predator being the focal species in the predator
to prey similarity (x-axis), while the prey is the focal one in the prey
to predator similarity (y-axis). One of the predators, \emph{Canis
aureus}, is not represented in the image because it is an extreme case
(where all its range is suppressed by the absence of preys) and it would
make the interpretation of the data more difficult.}\label{fig:geo_diss}
}
\end{figure}

There was high variation in the overlap of predator and prey ranges
(fig.~\ref{fig:geo_diss}). The high density of points on the left-hand
side of fig.~\ref{fig:geo_diss} indicates that most preys have small
ranges in comparison to those of the set of carnivores in the networks,
resulting in either low overlap between both ranges (bottom) or high
overlap of ranges because much of that of the prey is within predators'
range (top). The top-right side of the plot encompasses situations where
the ranges of both predator and prey are similar and overlapping, while
the bottom-right part of the plot represents a situation where the range
of the predator is smaller than that of its prey and much of it occurs
within the preys' range. For example, \emph{Panthera pardus} had many
preys occurring inside its range, with highly variable levels of overlap
(tbl.~\ref{tbl:everyone}). In general, species exhibited more consistent
values of prey-predator overlap, than predator-prey overlap -- indicated
by the spread of points along the x-axis, yet more restricted variation
on the y-axis (fig.~\ref{fig:geo_diss}). There was also no overall
relationship between the two metrics, or for any predator species.

\hypertarget{tbl:everyone}{}
\begin{longtable}[]{@{}lrrrrr@{}}
\caption{\label{tbl:everyone}List of species analyzed, their out and in
degrees, total original range size (in pixels), and proportion of their
ranges occupied by their preys and predators (values between 0 and 1).
Species are sorted according to the groups identified by Baskerville et
al. (2011). Notice how some species are isolated in the network
(\emph{Loxodonta africana}) and how \emph{Canis aureus}'s range does not
overlap with any of its preys.}\tabularnewline
\toprule
\begin{minipage}[b]{0.28\columnwidth}\raggedright
Species\strut
\end{minipage} & \begin{minipage}[b]{0.10\columnwidth}\raggedleft
Number of preys\strut
\end{minipage} & \begin{minipage}[b]{0.10\columnwidth}\raggedleft
Number of predators\strut
\end{minipage} & \begin{minipage}[b]{0.10\columnwidth}\raggedleft
Total range size\strut
\end{minipage} & \begin{minipage}[b]{0.13\columnwidth}\raggedleft
Proportion of range occupied by preys\strut
\end{minipage} & \begin{minipage}[b]{0.13\columnwidth}\raggedleft
Proportion of range occupied by predators\strut
\end{minipage}\tabularnewline
\midrule
\endfirsthead
\toprule
\begin{minipage}[b]{0.28\columnwidth}\raggedright
Species\strut
\end{minipage} & \begin{minipage}[b]{0.10\columnwidth}\raggedleft
Number of preys\strut
\end{minipage} & \begin{minipage}[b]{0.10\columnwidth}\raggedleft
Number of predators\strut
\end{minipage} & \begin{minipage}[b]{0.10\columnwidth}\raggedleft
Total range size\strut
\end{minipage} & \begin{minipage}[b]{0.13\columnwidth}\raggedleft
Proportion of range occupied by preys\strut
\end{minipage} & \begin{minipage}[b]{0.13\columnwidth}\raggedleft
Proportion of range occupied by predators\strut
\end{minipage}\tabularnewline
\midrule
\endhead
\begin{minipage}[t]{0.28\columnwidth}\raggedright
\textbf{Large carnivores}\strut
\end{minipage} & \begin{minipage}[t]{0.10\columnwidth}\raggedleft
\strut
\end{minipage} & \begin{minipage}[t]{0.10\columnwidth}\raggedleft
\strut
\end{minipage} & \begin{minipage}[t]{0.10\columnwidth}\raggedleft
\strut
\end{minipage} & \begin{minipage}[t]{0.13\columnwidth}\raggedleft
\strut
\end{minipage} & \begin{minipage}[t]{0.13\columnwidth}\raggedleft
\strut
\end{minipage}\tabularnewline
\begin{minipage}[t]{0.28\columnwidth}\raggedright
Acinonyx jubatus\strut
\end{minipage} & \begin{minipage}[t]{0.10\columnwidth}\raggedleft
8\strut
\end{minipage} & \begin{minipage}[t]{0.10\columnwidth}\raggedleft
1\strut
\end{minipage} & \begin{minipage}[t]{0.10\columnwidth}\raggedleft
15540\strut
\end{minipage} & \begin{minipage}[t]{0.13\columnwidth}\raggedleft
0.560\strut
\end{minipage} & \begin{minipage}[t]{0.13\columnwidth}\raggedleft
0.670\strut
\end{minipage}\tabularnewline
\begin{minipage}[t]{0.28\columnwidth}\raggedright
Crocuta crocuta\strut
\end{minipage} & \begin{minipage}[t]{0.10\columnwidth}\raggedleft
12\strut
\end{minipage} & \begin{minipage}[t]{0.10\columnwidth}\raggedleft
1\strut
\end{minipage} & \begin{minipage}[t]{0.10\columnwidth}\raggedleft
43307\strut
\end{minipage} & \begin{minipage}[t]{0.13\columnwidth}\raggedleft
0.848\strut
\end{minipage} & \begin{minipage}[t]{0.13\columnwidth}\raggedleft
0.252\strut
\end{minipage}\tabularnewline
\begin{minipage}[t]{0.28\columnwidth}\raggedright
Lycaon pictus\strut
\end{minipage} & \begin{minipage}[t]{0.10\columnwidth}\raggedleft
14\strut
\end{minipage} & \begin{minipage}[t]{0.10\columnwidth}\raggedleft
0\strut
\end{minipage} & \begin{minipage}[t]{0.10\columnwidth}\raggedleft
3873\strut
\end{minipage} & \begin{minipage}[t]{0.13\columnwidth}\raggedleft
0.916\strut
\end{minipage} & \begin{minipage}[t]{0.13\columnwidth}\raggedleft
-\strut
\end{minipage}\tabularnewline
\begin{minipage}[t]{0.28\columnwidth}\raggedright
Panthera leo\strut
\end{minipage} & \begin{minipage}[t]{0.10\columnwidth}\raggedleft
18\strut
\end{minipage} & \begin{minipage}[t]{0.10\columnwidth}\raggedleft
0\strut
\end{minipage} & \begin{minipage}[t]{0.10\columnwidth}\raggedleft
11384\strut
\end{minipage} & \begin{minipage}[t]{0.13\columnwidth}\raggedleft
0.934\strut
\end{minipage} & \begin{minipage}[t]{0.13\columnwidth}\raggedleft
-\strut
\end{minipage}\tabularnewline
\begin{minipage}[t]{0.28\columnwidth}\raggedright
Panthera pardus\strut
\end{minipage} & \begin{minipage}[t]{0.10\columnwidth}\raggedleft
22\strut
\end{minipage} & \begin{minipage}[t]{0.10\columnwidth}\raggedleft
0\strut
\end{minipage} & \begin{minipage}[t]{0.10\columnwidth}\raggedleft
68137\strut
\end{minipage} & \begin{minipage}[t]{0.13\columnwidth}\raggedleft
0.766\strut
\end{minipage} & \begin{minipage}[t]{0.13\columnwidth}\raggedleft
-\strut
\end{minipage}\tabularnewline
\begin{minipage}[t]{0.28\columnwidth}\raggedright
\strut
\end{minipage} & \begin{minipage}[t]{0.10\columnwidth}\raggedleft
\strut
\end{minipage} & \begin{minipage}[t]{0.10\columnwidth}\raggedleft
\strut
\end{minipage} & \begin{minipage}[t]{0.10\columnwidth}\raggedleft
\strut
\end{minipage} & \begin{minipage}[t]{0.13\columnwidth}\raggedleft
\strut
\end{minipage} & \begin{minipage}[t]{0.13\columnwidth}\raggedleft
\strut
\end{minipage}\tabularnewline
\begin{minipage}[t]{0.28\columnwidth}\raggedright
\textbf{Small carnivores}\strut
\end{minipage} & \begin{minipage}[t]{0.10\columnwidth}\raggedleft
\strut
\end{minipage} & \begin{minipage}[t]{0.10\columnwidth}\raggedleft
\strut
\end{minipage} & \begin{minipage}[t]{0.10\columnwidth}\raggedleft
\strut
\end{minipage} & \begin{minipage}[t]{0.13\columnwidth}\raggedleft
\strut
\end{minipage} & \begin{minipage}[t]{0.13\columnwidth}\raggedleft
\strut
\end{minipage}\tabularnewline
\begin{minipage}[t]{0.28\columnwidth}\raggedright
Canis aureus\strut
\end{minipage} & \begin{minipage}[t]{0.10\columnwidth}\raggedleft
4\strut
\end{minipage} & \begin{minipage}[t]{0.10\columnwidth}\raggedleft
1\strut
\end{minipage} & \begin{minipage}[t]{0.10\columnwidth}\raggedleft
7358\strut
\end{minipage} & \begin{minipage}[t]{0.13\columnwidth}\raggedleft
0.000\strut
\end{minipage} & \begin{minipage}[t]{0.13\columnwidth}\raggedleft
0.780\strut
\end{minipage}\tabularnewline
\begin{minipage}[t]{0.28\columnwidth}\raggedright
Canis mesomelas\strut
\end{minipage} & \begin{minipage}[t]{0.10\columnwidth}\raggedleft
1\strut
\end{minipage} & \begin{minipage}[t]{0.10\columnwidth}\raggedleft
1\strut
\end{minipage} & \begin{minipage}[t]{0.10\columnwidth}\raggedleft
19872\strut
\end{minipage} & \begin{minipage}[t]{0.13\columnwidth}\raggedleft
0.190\strut
\end{minipage} & \begin{minipage}[t]{0.13\columnwidth}\raggedleft
0.995\strut
\end{minipage}\tabularnewline
\begin{minipage}[t]{0.28\columnwidth}\raggedright
Caracal caracal\strut
\end{minipage} & \begin{minipage}[t]{0.10\columnwidth}\raggedleft
4\strut
\end{minipage} & \begin{minipage}[t]{0.10\columnwidth}\raggedleft
0\strut
\end{minipage} & \begin{minipage}[t]{0.10\columnwidth}\raggedleft
47243\strut
\end{minipage} & \begin{minipage}[t]{0.13\columnwidth}\raggedleft
0.832\strut
\end{minipage} & \begin{minipage}[t]{0.13\columnwidth}\raggedleft
-\strut
\end{minipage}\tabularnewline
\begin{minipage}[t]{0.28\columnwidth}\raggedright
Leptailurus serval\strut
\end{minipage} & \begin{minipage}[t]{0.10\columnwidth}\raggedleft
1\strut
\end{minipage} & \begin{minipage}[t]{0.10\columnwidth}\raggedleft
1\strut
\end{minipage} & \begin{minipage}[t]{0.10\columnwidth}\raggedleft
38856\strut
\end{minipage} & \begin{minipage}[t]{0.13\columnwidth}\raggedleft
0.011\strut
\end{minipage} & \begin{minipage}[t]{0.13\columnwidth}\raggedleft
0.979\strut
\end{minipage}\tabularnewline
\begin{minipage}[t]{0.28\columnwidth}\raggedright
\strut
\end{minipage} & \begin{minipage}[t]{0.10\columnwidth}\raggedleft
\strut
\end{minipage} & \begin{minipage}[t]{0.10\columnwidth}\raggedleft
\strut
\end{minipage} & \begin{minipage}[t]{0.10\columnwidth}\raggedleft
\strut
\end{minipage} & \begin{minipage}[t]{0.13\columnwidth}\raggedleft
\strut
\end{minipage} & \begin{minipage}[t]{0.13\columnwidth}\raggedleft
\strut
\end{minipage}\tabularnewline
\begin{minipage}[t]{0.28\columnwidth}\raggedright
\textbf{Small herbivores}\strut
\end{minipage} & \begin{minipage}[t]{0.10\columnwidth}\raggedleft
\strut
\end{minipage} & \begin{minipage}[t]{0.10\columnwidth}\raggedleft
\strut
\end{minipage} & \begin{minipage}[t]{0.10\columnwidth}\raggedleft
\strut
\end{minipage} & \begin{minipage}[t]{0.13\columnwidth}\raggedleft
\strut
\end{minipage} & \begin{minipage}[t]{0.13\columnwidth}\raggedleft
\strut
\end{minipage}\tabularnewline
\begin{minipage}[t]{0.28\columnwidth}\raggedright
Damaliscus lunatus\strut
\end{minipage} & \begin{minipage}[t]{0.10\columnwidth}\raggedleft
0\strut
\end{minipage} & \begin{minipage}[t]{0.10\columnwidth}\raggedleft
4\strut
\end{minipage} & \begin{minipage}[t]{0.10\columnwidth}\raggedleft
5567\strut
\end{minipage} & \begin{minipage}[t]{0.13\columnwidth}\raggedleft
-\strut
\end{minipage} & \begin{minipage}[t]{0.13\columnwidth}\raggedleft
1\strut
\end{minipage}\tabularnewline
\begin{minipage}[t]{0.28\columnwidth}\raggedright
Hippopotamus amphibius\strut
\end{minipage} & \begin{minipage}[t]{0.10\columnwidth}\raggedleft
0\strut
\end{minipage} & \begin{minipage}[t]{0.10\columnwidth}\raggedleft
0\strut
\end{minipage} & \begin{minipage}[t]{0.10\columnwidth}\raggedleft
3695\strut
\end{minipage} & \begin{minipage}[t]{0.13\columnwidth}\raggedleft
-\strut
\end{minipage} & \begin{minipage}[t]{0.13\columnwidth}\raggedleft
-\strut
\end{minipage}\tabularnewline
\begin{minipage}[t]{0.28\columnwidth}\raggedright
Kobus ellipsiprymnus\strut
\end{minipage} & \begin{minipage}[t]{0.10\columnwidth}\raggedleft
0\strut
\end{minipage} & \begin{minipage}[t]{0.10\columnwidth}\raggedleft
4\strut
\end{minipage} & \begin{minipage}[t]{0.10\columnwidth}\raggedleft
26705\strut
\end{minipage} & \begin{minipage}[t]{0.13\columnwidth}\raggedleft
-\strut
\end{minipage} & \begin{minipage}[t]{0.13\columnwidth}\raggedleft
1\strut
\end{minipage}\tabularnewline
\begin{minipage}[t]{0.28\columnwidth}\raggedright
Ourebia ourebi\strut
\end{minipage} & \begin{minipage}[t]{0.10\columnwidth}\raggedleft
0\strut
\end{minipage} & \begin{minipage}[t]{0.10\columnwidth}\raggedleft
5\strut
\end{minipage} & \begin{minipage}[t]{0.10\columnwidth}\raggedleft
22380\strut
\end{minipage} & \begin{minipage}[t]{0.13\columnwidth}\raggedleft
-\strut
\end{minipage} & \begin{minipage}[t]{0.13\columnwidth}\raggedleft
1\strut
\end{minipage}\tabularnewline
\begin{minipage}[t]{0.28\columnwidth}\raggedright
Pedetes capensis\strut
\end{minipage} & \begin{minipage}[t]{0.10\columnwidth}\raggedleft
0\strut
\end{minipage} & \begin{minipage}[t]{0.10\columnwidth}\raggedleft
2\strut
\end{minipage} & \begin{minipage}[t]{0.10\columnwidth}\raggedleft
11901\strut
\end{minipage} & \begin{minipage}[t]{0.13\columnwidth}\raggedleft
-\strut
\end{minipage} & \begin{minipage}[t]{0.13\columnwidth}\raggedleft
1\strut
\end{minipage}\tabularnewline
\begin{minipage}[t]{0.28\columnwidth}\raggedright
Phacochoerus africanus\strut
\end{minipage} & \begin{minipage}[t]{0.10\columnwidth}\raggedleft
0\strut
\end{minipage} & \begin{minipage}[t]{0.10\columnwidth}\raggedleft
5\strut
\end{minipage} & \begin{minipage}[t]{0.10\columnwidth}\raggedleft
29963\strut
\end{minipage} & \begin{minipage}[t]{0.13\columnwidth}\raggedleft
-\strut
\end{minipage} & \begin{minipage}[t]{0.13\columnwidth}\raggedleft
0.999\strut
\end{minipage}\tabularnewline
\begin{minipage}[t]{0.28\columnwidth}\raggedright
Redunca redunca\strut
\end{minipage} & \begin{minipage}[t]{0.10\columnwidth}\raggedleft
0\strut
\end{minipage} & \begin{minipage}[t]{0.10\columnwidth}\raggedleft
5\strut
\end{minipage} & \begin{minipage}[t]{0.10\columnwidth}\raggedleft
17465\strut
\end{minipage} & \begin{minipage}[t]{0.13\columnwidth}\raggedleft
-\strut
\end{minipage} & \begin{minipage}[t]{0.13\columnwidth}\raggedleft
1\strut
\end{minipage}\tabularnewline
\begin{minipage}[t]{0.28\columnwidth}\raggedright
Rhabdomys pumilio\strut
\end{minipage} & \begin{minipage}[t]{0.10\columnwidth}\raggedleft
0\strut
\end{minipage} & \begin{minipage}[t]{0.10\columnwidth}\raggedleft
5\strut
\end{minipage} & \begin{minipage}[t]{0.10\columnwidth}\raggedleft
465\strut
\end{minipage} & \begin{minipage}[t]{0.13\columnwidth}\raggedleft
-\strut
\end{minipage} & \begin{minipage}[t]{0.13\columnwidth}\raggedleft
0.998\strut
\end{minipage}\tabularnewline
\begin{minipage}[t]{0.28\columnwidth}\raggedright
Tragelaphus oryx\strut
\end{minipage} & \begin{minipage}[t]{0.10\columnwidth}\raggedleft
0\strut
\end{minipage} & \begin{minipage}[t]{0.10\columnwidth}\raggedleft
2\strut
\end{minipage} & \begin{minipage}[t]{0.10\columnwidth}\raggedleft
20852\strut
\end{minipage} & \begin{minipage}[t]{0.13\columnwidth}\raggedleft
-\strut
\end{minipage} & \begin{minipage}[t]{0.13\columnwidth}\raggedleft
0.991\strut
\end{minipage}\tabularnewline
\begin{minipage}[t]{0.28\columnwidth}\raggedright
Tragelaphus scriptus\strut
\end{minipage} & \begin{minipage}[t]{0.10\columnwidth}\raggedleft
0\strut
\end{minipage} & \begin{minipage}[t]{0.10\columnwidth}\raggedleft
3\strut
\end{minipage} & \begin{minipage}[t]{0.10\columnwidth}\raggedleft
36011\strut
\end{minipage} & \begin{minipage}[t]{0.13\columnwidth}\raggedleft
-\strut
\end{minipage} & \begin{minipage}[t]{0.13\columnwidth}\raggedleft
0.984\strut
\end{minipage}\tabularnewline
\begin{minipage}[t]{0.28\columnwidth}\raggedright
\strut
\end{minipage} & \begin{minipage}[t]{0.10\columnwidth}\raggedleft
\strut
\end{minipage} & \begin{minipage}[t]{0.10\columnwidth}\raggedleft
\strut
\end{minipage} & \begin{minipage}[t]{0.10\columnwidth}\raggedleft
\strut
\end{minipage} & \begin{minipage}[t]{0.13\columnwidth}\raggedleft
\strut
\end{minipage} & \begin{minipage}[t]{0.13\columnwidth}\raggedleft
\strut
\end{minipage}\tabularnewline
\begin{minipage}[t]{0.28\columnwidth}\raggedright
\textbf{Large grazers}\strut
\end{minipage} & \begin{minipage}[t]{0.10\columnwidth}\raggedleft
\strut
\end{minipage} & \begin{minipage}[t]{0.10\columnwidth}\raggedleft
\strut
\end{minipage} & \begin{minipage}[t]{0.10\columnwidth}\raggedleft
\strut
\end{minipage} & \begin{minipage}[t]{0.13\columnwidth}\raggedleft
\strut
\end{minipage} & \begin{minipage}[t]{0.13\columnwidth}\raggedleft
\strut
\end{minipage}\tabularnewline
\begin{minipage}[t]{0.28\columnwidth}\raggedright
Aepyceros melampus\strut
\end{minipage} & \begin{minipage}[t]{0.10\columnwidth}\raggedleft
0\strut
\end{minipage} & \begin{minipage}[t]{0.10\columnwidth}\raggedleft
5\strut
\end{minipage} & \begin{minipage}[t]{0.10\columnwidth}\raggedleft
10579\strut
\end{minipage} & \begin{minipage}[t]{0.13\columnwidth}\raggedleft
-\strut
\end{minipage} & \begin{minipage}[t]{0.13\columnwidth}\raggedleft
1\strut
\end{minipage}\tabularnewline
\begin{minipage}[t]{0.28\columnwidth}\raggedright
Alcelaphus buselaphus\strut
\end{minipage} & \begin{minipage}[t]{0.10\columnwidth}\raggedleft
0\strut
\end{minipage} & \begin{minipage}[t]{0.10\columnwidth}\raggedleft
4\strut
\end{minipage} & \begin{minipage}[t]{0.10\columnwidth}\raggedleft
20761\strut
\end{minipage} & \begin{minipage}[t]{0.13\columnwidth}\raggedleft
-\strut
\end{minipage} & \begin{minipage}[t]{0.13\columnwidth}\raggedleft
1\strut
\end{minipage}\tabularnewline
\begin{minipage}[t]{0.28\columnwidth}\raggedright
Connochaetes taurinus\strut
\end{minipage} & \begin{minipage}[t]{0.10\columnwidth}\raggedleft
0\strut
\end{minipage} & \begin{minipage}[t]{0.10\columnwidth}\raggedleft
6\strut
\end{minipage} & \begin{minipage}[t]{0.10\columnwidth}\raggedleft
9650\strut
\end{minipage} & \begin{minipage}[t]{0.13\columnwidth}\raggedleft
-\strut
\end{minipage} & \begin{minipage}[t]{0.13\columnwidth}\raggedleft
1\strut
\end{minipage}\tabularnewline
\begin{minipage}[t]{0.28\columnwidth}\raggedright
Equus quagga\strut
\end{minipage} & \begin{minipage}[t]{0.10\columnwidth}\raggedleft
0\strut
\end{minipage} & \begin{minipage}[t]{0.10\columnwidth}\raggedleft
5\strut
\end{minipage} & \begin{minipage}[t]{0.10\columnwidth}\raggedleft
7070\strut
\end{minipage} & \begin{minipage}[t]{0.13\columnwidth}\raggedleft
-\strut
\end{minipage} & \begin{minipage}[t]{0.13\columnwidth}\raggedleft
1\strut
\end{minipage}\tabularnewline
\begin{minipage}[t]{0.28\columnwidth}\raggedright
Eudorcas thomsonii\strut
\end{minipage} & \begin{minipage}[t]{0.10\columnwidth}\raggedleft
0\strut
\end{minipage} & \begin{minipage}[t]{0.10\columnwidth}\raggedleft
6\strut
\end{minipage} & \begin{minipage}[t]{0.10\columnwidth}\raggedleft
463\strut
\end{minipage} & \begin{minipage}[t]{0.13\columnwidth}\raggedleft
-\strut
\end{minipage} & \begin{minipage}[t]{0.13\columnwidth}\raggedleft
1\strut
\end{minipage}\tabularnewline
\begin{minipage}[t]{0.28\columnwidth}\raggedright
Nanger granti\strut
\end{minipage} & \begin{minipage}[t]{0.10\columnwidth}\raggedleft
0\strut
\end{minipage} & \begin{minipage}[t]{0.10\columnwidth}\raggedleft
6\strut
\end{minipage} & \begin{minipage}[t]{0.10\columnwidth}\raggedleft
2303\strut
\end{minipage} & \begin{minipage}[t]{0.13\columnwidth}\raggedleft
-\strut
\end{minipage} & \begin{minipage}[t]{0.13\columnwidth}\raggedleft
1\strut
\end{minipage}\tabularnewline
\begin{minipage}[t]{0.28\columnwidth}\raggedright
\strut
\end{minipage} & \begin{minipage}[t]{0.10\columnwidth}\raggedleft
\strut
\end{minipage} & \begin{minipage}[t]{0.10\columnwidth}\raggedleft
\strut
\end{minipage} & \begin{minipage}[t]{0.10\columnwidth}\raggedleft
\strut
\end{minipage} & \begin{minipage}[t]{0.13\columnwidth}\raggedleft
\strut
\end{minipage} & \begin{minipage}[t]{0.13\columnwidth}\raggedleft
\strut
\end{minipage}\tabularnewline
\begin{minipage}[t]{0.28\columnwidth}\raggedright
\textbf{Hyraxes}\strut
\end{minipage} & \begin{minipage}[t]{0.10\columnwidth}\raggedleft
\strut
\end{minipage} & \begin{minipage}[t]{0.10\columnwidth}\raggedleft
\strut
\end{minipage} & \begin{minipage}[t]{0.10\columnwidth}\raggedleft
\strut
\end{minipage} & \begin{minipage}[t]{0.13\columnwidth}\raggedleft
\strut
\end{minipage} & \begin{minipage}[t]{0.13\columnwidth}\raggedleft
\strut
\end{minipage}\tabularnewline
\begin{minipage}[t]{0.28\columnwidth}\raggedright
Heterohyrax brucei\strut
\end{minipage} & \begin{minipage}[t]{0.10\columnwidth}\raggedleft
0\strut
\end{minipage} & \begin{minipage}[t]{0.10\columnwidth}\raggedleft
1\strut
\end{minipage} & \begin{minipage}[t]{0.10\columnwidth}\raggedleft
17728\strut
\end{minipage} & \begin{minipage}[t]{0.13\columnwidth}\raggedleft
-\strut
\end{minipage} & \begin{minipage}[t]{0.13\columnwidth}\raggedleft
0.972\strut
\end{minipage}\tabularnewline
\begin{minipage}[t]{0.28\columnwidth}\raggedright
Procavia capensis\strut
\end{minipage} & \begin{minipage}[t]{0.10\columnwidth}\raggedleft
0\strut
\end{minipage} & \begin{minipage}[t]{0.10\columnwidth}\raggedleft
1\strut
\end{minipage} & \begin{minipage}[t]{0.10\columnwidth}\raggedleft
47697\strut
\end{minipage} & \begin{minipage}[t]{0.13\columnwidth}\raggedleft
-\strut
\end{minipage} & \begin{minipage}[t]{0.13\columnwidth}\raggedleft
0.647\strut
\end{minipage}\tabularnewline
\begin{minipage}[t]{0.28\columnwidth}\raggedright
\strut
\end{minipage} & \begin{minipage}[t]{0.10\columnwidth}\raggedleft
\strut
\end{minipage} & \begin{minipage}[t]{0.10\columnwidth}\raggedleft
\strut
\end{minipage} & \begin{minipage}[t]{0.10\columnwidth}\raggedleft
\strut
\end{minipage} & \begin{minipage}[t]{0.13\columnwidth}\raggedleft
\strut
\end{minipage} & \begin{minipage}[t]{0.13\columnwidth}\raggedleft
\strut
\end{minipage}\tabularnewline
\begin{minipage}[t]{0.28\columnwidth}\raggedright
\textbf{Others}\strut
\end{minipage} & \begin{minipage}[t]{0.10\columnwidth}\raggedleft
\strut
\end{minipage} & \begin{minipage}[t]{0.10\columnwidth}\raggedleft
\strut
\end{minipage} & \begin{minipage}[t]{0.10\columnwidth}\raggedleft
\strut
\end{minipage} & \begin{minipage}[t]{0.13\columnwidth}\raggedleft
\strut
\end{minipage} & \begin{minipage}[t]{0.13\columnwidth}\raggedleft
\strut
\end{minipage}\tabularnewline
\begin{minipage}[t]{0.28\columnwidth}\raggedright
Giraffa camelopardalis\strut
\end{minipage} & \begin{minipage}[t]{0.10\columnwidth}\raggedleft
0\strut
\end{minipage} & \begin{minipage}[t]{0.10\columnwidth}\raggedleft
1\strut
\end{minipage} & \begin{minipage}[t]{0.10\columnwidth}\raggedleft
5418\strut
\end{minipage} & \begin{minipage}[t]{0.13\columnwidth}\raggedleft
-\strut
\end{minipage} & \begin{minipage}[t]{0.13\columnwidth}\raggedleft
0.470\strut
\end{minipage}\tabularnewline
\begin{minipage}[t]{0.28\columnwidth}\raggedright
Loxodonta africana\strut
\end{minipage} & \begin{minipage}[t]{0.10\columnwidth}\raggedleft
0\strut
\end{minipage} & \begin{minipage}[t]{0.10\columnwidth}\raggedleft
0\strut
\end{minipage} & \begin{minipage}[t]{0.10\columnwidth}\raggedleft
9654\strut
\end{minipage} & \begin{minipage}[t]{0.13\columnwidth}\raggedleft
-\strut
\end{minipage} & \begin{minipage}[t]{0.13\columnwidth}\raggedleft
-\strut
\end{minipage}\tabularnewline
\begin{minipage}[t]{0.28\columnwidth}\raggedright
Madoqua kirkii\strut
\end{minipage} & \begin{minipage}[t]{0.10\columnwidth}\raggedleft
0\strut
\end{minipage} & \begin{minipage}[t]{0.10\columnwidth}\raggedleft
7\strut
\end{minipage} & \begin{minipage}[t]{0.10\columnwidth}\raggedleft
4002\strut
\end{minipage} & \begin{minipage}[t]{0.13\columnwidth}\raggedleft
-\strut
\end{minipage} & \begin{minipage}[t]{0.13\columnwidth}\raggedleft
1\strut
\end{minipage}\tabularnewline
\begin{minipage}[t]{0.28\columnwidth}\raggedright
Papio anubis\strut
\end{minipage} & \begin{minipage}[t]{0.10\columnwidth}\raggedleft
0\strut
\end{minipage} & \begin{minipage}[t]{0.10\columnwidth}\raggedleft
1\strut
\end{minipage} & \begin{minipage}[t]{0.10\columnwidth}\raggedleft
23171\strut
\end{minipage} & \begin{minipage}[t]{0.13\columnwidth}\raggedleft
-\strut
\end{minipage} & \begin{minipage}[t]{0.13\columnwidth}\raggedleft
0.938\strut
\end{minipage}\tabularnewline
\begin{minipage}[t]{0.28\columnwidth}\raggedright
Syncerus caffer\strut
\end{minipage} & \begin{minipage}[t]{0.10\columnwidth}\raggedleft
0\strut
\end{minipage} & \begin{minipage}[t]{0.10\columnwidth}\raggedleft
1\strut
\end{minipage} & \begin{minipage}[t]{0.10\columnwidth}\raggedleft
25223\strut
\end{minipage} & \begin{minipage}[t]{0.13\columnwidth}\raggedleft
-\strut
\end{minipage} & \begin{minipage}[t]{0.13\columnwidth}\raggedleft
0.250\strut
\end{minipage}\tabularnewline
\bottomrule
\end{longtable}

\hypertarget{validation-with-gbif-occurrences}{%
\subsection{Validation with GBIF
occurrences}\label{validation-with-gbif-occurrences}}

The proportion of GBIF pixels (pixels with at least one GBIF occurrence)
falling in the IUCN ranges varied from low to high depending on the
species (fig.~\ref{fig:gbif}, left). The lowest proportions occurred for
species with small ranges (such as \emph{Lycaon pictus}), although some
species with small ranges showed high overlap. Species with median and
large ranges had high proportions of occurrences falling into their IUCN
range. Predators and preys displayed similar overlap variations. While
no species had all of its GBIF occurrences within its IUCN range, one
species had this proportion equal to zero, \emph{Canis aureus}, which is
also the only species whose range is not covered by any of its preys.
This result reinforces the concern raised in the literature on the use
of IUCN range maps for species that are not well known (Herkt, Skidmore,
and Fahr 2017), demonstrating how small range species are likely to have
their distribution underestimated in the IUCN database. Additionally,
the fact that \emph{Canis aureus} had none of its GBIF pixels
overlapping with IUCN maps suggests a taxonomic mismatch between both
databases, which we explore in the Discussion section.

The proportion of GBIF pixels in updated ranges can only be equal to or
lower than that of the original ranges, as our analysis removes pixels
from the original range and does not add new ones. Rather, the absence
of a difference between the two types of ranges indicates that no pixels
with GBIF observations, hence likely true habitats, were removed by our
analysis. Here this proportion was mostly similar to that of the
original IUCN ranges for most predator species (fig.~\ref{fig:gbif}).
Four species showed no difference in proportion while three species
showed only small differences (proportions of 0.01 to 0.05). On the
other hand, two species, \emph{Canis mesomelas}, and \emph{Leptailurus
serval} showed very high differences, with overlaps lower by 0.548 and
0.871 respectively. For \emph{Leptailurus serval}, none of the GBIF
observations occurred in the updated range. These two species are also
the only predators with a single prey in our metaweb. Our results
delineate how a mismatch between GBIF and IUCN databases differ greatly
with small changes in herbivore species ranges, and it is somewhat
positively related to range size for predator species. Moreover, we show
that accounting for interactions does not necessarily aggravates this
dissimilarity, but it is relevant for species with little ecological
information or specialists.

\begin{figure}
\hypertarget{fig:gbif}{%
\centering
\includegraphics{figures/gbif_panels.png}
\caption{Left panel: Distribution of the proportion of GBIF pixels
(pixels with at least one occurrence in GBIF) falling into the IUCN
range for different range sizes. Right panel: Differences between the
proportion of GBIF pixels falling into the IUCN and the updated ranges
for every predator species. Arrows go from the proportion inside the
original range to the proportion inside the updated range, which can
only be equal or lower. Overlapping markers indicate no difference
between the types of layers. Species markers are the same on both
figures, with predators presented in distinct colored markers and all
herbivores grouped in a single grey marker. Pixels represent a
resolution of 10 arc-minutes.}\label{fig:gbif}
}
\end{figure}

\hypertarget{discussion}{%
\section{Discussion}\label{discussion}}

The jackal is a widespread taxon in northern Africa, Europe, and
Australasia, generally well adapted to local conditions due to its
largely varied diet (Tsunoda and Saito 2020; Krofel et al. 2021).
Because of that, we expected that the \emph{Canis} species in our
dataset would be the ones losing the least amount of range, with a
higher value of the proportion of GBIF pixels within their IUCN range
maps. However, the taxonomy of this group is a matter of intense
discussion, as molecular and morphological data seem to disagree in the
clustering of species and subspecies (Krofel et al. 2021; Stoyanov
2020). This debate is indeed reflected in our analysis: the GBIF
identification of the golden jackal is incompatible with the one used by
IUCN, each of them mapping its distribution in completely different
places. This led to a complete exclusion of \emph{Canis aureus} from its
original range in our analysis, despite the fact that this species has
four documented preys in our metaweb. This example illustrates how the
taxonomic, geographical and ecological data can be used to validate one
another.

Here we show that when ecological interaction data (predator-prey
interactions within food webs) are used to refine species range maps,
there are significant reductions in the IUCN range size of predatory
organisms. Despite showing the potential importance of accounting for
species interactions when estimating the range of a species, it remains
unclear the extent to which the patterns observed represent ecological
processes or a lack of data. In the following sections, we discuss the
implications of our findings, in terms of species range maps,
interaction data, and the next steps required to enhance understanding
of species distributions using information on ecological networks.

\hypertarget{connectivity-diversity-and-range-preservation}{%
\subsection{Connectivity, diversity and range
preservation}\label{connectivity-diversity-and-range-preservation}}

In the Serengeti food web there is a positive relationship between the
out degrees of predators and the size of their ranges
(tbl.~\ref{tbl:everyone}). In addition, our results show that there is a
negative relationship between the relative loss of predators' ranges and
their number of preys (fig.~\ref{fig:degree}), reinforcing the idea that
generalist species can preserve their distributions longer while losing
interactions. The factors limiting the geographical range of a species
in a community can vary with connectivity and richness (Svenning et al.
2014). Younger communities may be more affected by environmental
limitations because they are dominated by generalist species, while
older metacommunities are probably affected in different ways in the
center of the distribution, at the edge of ranges, and in sink and
source communities (Svenning et al. 2014; Godsoe et al. 2017; Cazelles
et al. 2016; Bullock et al. 2000). Additionally, it is likely that
species with larger ranges of distribution and those that are more
generalists would co-occur with a greater number of other species
(Dáttilo et al. 2020), while dispersal capacity of competitive species
modulate their aggregation in space and the effect of interactions on
their range limits (Godsoe et al. 2017).

\hypertarget{geographical-mismatch-and-data-availability}{%
\subsection{Geographical mismatch and data
availability}\label{geographical-mismatch-and-data-availability}}

The geographical mismatch between predators and preys has ecological
consequences such as loss of ecosystem functioning and extinction of
populations (Anderson et al. 2016; Dáttilo and Rico-Gray 2018; Pringle
et al. 2016; Young et al. 2013). Climate change is one of the causes of
this, leading, for instance, to the decrease of plant populations due to
the lack of pollination (Bullock et al. 2000; Afkhami, McIntyre, and
Strauss 2014; Godsoe et al. 2017). However, this mismatch can also be
purely informational. When the distribution of predators and preys does
not superpose, it can mean we lack information about the distribution of
either species or about their interactions (e.g., predators may be
feeding on different species than the ones in our dataset outside the
Serengeti ecosystem). Here we addressed part of this problem by
comparing the IUCN range maps with GBIF occurrences, which helped us
clarify what is the shortfall for each species.

The lack of superposition between IUCN range maps and GBIF occurrences
suggests that we certainly do miss geographical information about the
distribution of a certain species, but it is not an indicator of the
completeness of the information about ecological interactions. However,
if both GBIF and IUCN occurrences tend to superpose and still the
species is locally removed, this indicates we don't have information
about all its interactions. The combination of this rationale with our
method of updating range maps based on ecological interactions allows us
to have a clearer idea of which information we are missing. For example,
the lion (\emph{Panthera leo}) was one of the species with the smallest
difference between the original and the updated ranges
(fig.~\ref{fig:degree}), but 59.5\% of the GBIF occurrences for this
species fell outside the IUCN range (fig.~\ref{fig:gbif}). In this
particular case, the IUCN maps seem to agree with species interaction
data. However, the disagreement between the IUCN and the GBIF databases
is concerning and suggests that the IUCN maps might underestimate the
lion's distribution. On the other hand, \emph{Leptailurus serval} and
\emph{Canis mesomelas} are two of the three species that lose the higher
proportion of range due to the lack of paths to a herbivore
(fig.~\ref{fig:degree}), but are also some of the species with the
higher proportion of GBIF occurrences inside IUCN range maps
(fig.~\ref{fig:gbif}). This indicates that the information we are
missing for these two species is related to either the occurrence of an
interaction or the presence of interacting species. To illustrate that,
we mapped the GBIF data for the prey of \emph{Leptailurus serval}, with
a mobility buffer around each point (fig.~\ref{fig:serval}). When
considering GBIF data, approximately 53\% of the prey's occurrences are
within the portion of the serval's range that was lost. With the buffer
area, this corresponds to 13\% of the lost range. This means that by
adding GBIF information, we would reduce the loss of range (or
information) for the predator by 13\% since its distribution is
conditional on the occurrence of its preys.

\begin{figure}
\hypertarget{fig:serval}{%
\centering
\includegraphics{figures/serval_mismatch_combined.png}
\caption{Mismatch between serval's range loss and GBIF occurrence of its
prey. The left panel shows the reduction of serval's range when we
consider the IUCN data on its prey. On the right panel, we added GBIF
data on both serval and its prey, with a buffer for the prey to account
for species mobility.}\label{fig:serval}
}
\end{figure}

Finally, the extreme case of \emph{Canis aureus} illustrates a lack of
both geographical and ecological information: none of its GBIF
occurrences and none of its preys occur inside its IUCN range. We
believe, therefore, that the validation of species distribution based on
ecological interaction is a relevant method that can further fill in
information gaps. Nevertheless, it is imperative that more
geographically explicit data about ecological networks and interactions
become available. This would help clarify when cooccurrences can be
translated into interactions and help the development of more advanced
validation methods for occurrence data.

\hypertarget{next-steps}{%
\subsection{Next steps}\label{next-steps}}

Here we demonstrated how we can detect uncertainty in species
distribution data using ecological interactions. Knowing where
questionable occurrence data are can be crucial in ecological modelling
(Hortal 2008; Ladle and Hortal 2013), and accounting for these errors
can improve model outputs by diminishing the error propagation (Draper
1995). For instance, we believe this is a way to account for ecological
interactions in habitat suitability models without making the models
more complex, but making sure (not assuming) that the input data - the
species occurrence - actually accounts for ecological interactions. It
is important to notice, however, that the quality and usefulness of this
method are highly correlated with the amount and quality of data
available about species' occurrences \textbf{and} interactions. With
this paper we hope to add to the collective effort to decode the
encrypted message that is the occurrence of a species in space and time.
A promising avenue that adds to our method is the prediction of networks
and interactions in large scales (Strydom et al. 2021), for they can add
valuable information about ecological interactions where they are
missing. Additionally, in order to achieve a robust modelling framework
towards actual species distribution models we should invest in efforts
to collect and combine open data on species occurrence and interactions,
especially because we may be losing ecological interactions at least as
fast as we are losing species (Valiente-Banuet et al. 2015).

\hypertarget{acknowledgements}{%
\section{Acknowledgements}\label{acknowledgements}}

We acknowledge that this study was conducted on land within the
traditional unceded territory of the Saint Lawrence Iroquoian,
Anishinabewaki, Mohawk, Huron-Wendat, and Omàmiwininiwak nations. GH,
FB, GD, and NF are funded by the NSERC Computational Biodiversity
Science and Services (BIOS\(^2\)) CREATE program; FB, NF, and TP are
funded by the Institute for Data Valorization (IVADO); NF and TP are
funded by a donation from the Courtois Foundation; GD is funded by the
FRQNT doctoral scholarship; TP is funded by the Canadian Institute of
Ecology \& Evolution; FW is funded by the Royal Society (Grant number:
CHL\textbackslash R1\textbackslash180156).

\hypertarget{references}{%
\section*{References}\label{references}}
\addcontentsline{toc}{section}{References}

\hypertarget{refs}{}
\begin{CSLReferences}{1}{0}
\leavevmode\hypertarget{ref-AbregoAccSpe}{}%
Abrego, Nerea, Tomas Roslin, Tea Huotari, Yinqiu Ji, Niels Martin
Schmidt, Jiaxin Wang, Douglas W. Yu, and Otso Ovaskainen. n.d.
{``Accounting for Species Interactions Is Necessary for Predicting How
Arctic Arthropod Communities Respond to Climate Change.''}
\emph{Ecography} n/a (n/a). \url{https://doi.org/10.1111/ecog.05547}.

\leavevmode\hypertarget{ref-Afkhami2014MutEff}{}%
Afkhami, Michelle E., Patrick J. McIntyre, and Sharon Y. Strauss. 2014.
{``Mutualist-Mediated Effects on Species' Range Limits Across Large
Geographic Scales.''} \emph{Ecology Letters} 17 (10): 1265--73.
\url{https://doi.org/10.1111/ele.12332}.

\leavevmode\hypertarget{ref-Albrecht2018PlaAni}{}%
Albrecht, Jörg. 2018. {``Plant and Animal Functional Diversity Drive
Mutualistic Network Assembly Across an Elevational Gradient.''}
\emph{NATURE COMMUNICATIONS}, 10.

\leavevmode\hypertarget{ref-Alhajeri2019HigCor}{}%
Alhajeri, Bader H, and Yoan Fourcade. 2019. {``High Correlation Between
Species-Level Environmental Data Estimates Extracted from IUCN Expert
Range Maps and from GBIF Occurrence Data.''} \emph{Journal of
Biogeography}, 13. \url{https://doi.org/10.1111/jbi.13619}.

\leavevmode\hypertarget{ref-Anderson2016SpaDis}{}%
Anderson, T. Michael, Staci White, Bryant Davis, Rob Erhardt, Meredith
Palmer, Alexandra Swanson, Margaret Kosmala, and Craig Packer. 2016.
{``The Spatial Distribution of African Savannah Herbivores: Species
Associations and Habitat Occupancy in a Landscape Context.''}
\emph{Philosophical Transactions of the Royal Society B: Biological
Sciences} 371 (1703): 20150314.
\url{https://doi.org/10.1098/rstb.2015.0314}.

\leavevmode\hypertarget{ref-Araujo2014ImpBio}{}%
Araújo, Carlos B. de, Luiz Octavio Marcondes-Machado, and Gabriel C.
Costa. 2014. {``The Importance of Biotic Interactions in Species
Distribution Models: A Test of the Eltonian Noise Hypothesis Using
Parrots.''} \emph{Journal of Biogeography} 41 (3): 513--23.
\url{https://doi.org/10.1111/jbi.12234}.

\leavevmode\hypertarget{ref-Baskerville2011SpaGui}{}%
Baskerville, Edward B., Andy P. Dobson, Trevor Bedford, Stefano
Allesina, T. Michael Anderson, and Mercedes Pascual. 2011. {``Spatial
Guilds in the Serengeti Food Web Revealed by a Bayesian Group Model.''}
\emph{PLOS Computational Biology} 7 (12): e1002321.
\url{https://doi.org/10.1371/journal.pcbi.1002321}.

\leavevmode\hypertarget{ref-Blanchet2020CooNot}{}%
Blanchet, F. Guillaume, Kevin Cazelles, and Dominique Gravel. 2020.
{``Co-Occurrence Is Not Evidence of Ecological Interactions.''}
\emph{Ecology Letters} 23 (7): 1050--63.
\url{https://doi.org/10.1111/ele.13525}.

\leavevmode\hypertarget{ref-Boakes2010DisVie}{}%
Boakes, Elizabeth H., Philip J. K. McGowan, Richard A. Fuller, Ding
Chang-qing, Natalie E. Clark, Kim O'Connor, and Georgina M. Mace. 2010.
{``Distorted Views of Biodiversity: Spatial and Temporal Bias in Species
Occurrence Data.''} \emph{PLOS Biology} 8 (6): e1000385.
\url{https://doi.org/10.1371/journal.pbio.1000385}.

\leavevmode\hypertarget{ref-Bullock2000GeoSep}{}%
Bullock, James M., Rebecca J. Edwards, Peter D. Carey, and Rob J. Rose.
2000. {``Geographical Separation of Two Ulex Species at Three Spatial
Scales: Does Competition Limit Species' Ranges?''} \emph{Ecography} 23
(2): 257--71. \url{https://doi.org/10.1111/j.1600-0587.2000.tb00281.x}.

\leavevmode\hypertarget{ref-Cabral2017MecSim}{}%
Cabral, Juliano Sarmento, Luis Valente, and Florian Hartig. 2017.
{``Mechanistic Simulation Models in Macroecology and Biogeography:
State-of-Art and Prospects.''} \emph{Ecography} 40 (2): 267--80.
\url{https://doi.org/10.1111/ecog.02480}.

\leavevmode\hypertarget{ref-Callaghan2019ImpBig}{}%
Callaghan, Corey T., Jodi J. L. Rowley, William K. Cornwell, Alistair G.
B. Poore, and Richard E. Major. 2019. {``Improving Big Citizen Science
Data: Moving Beyond Haphazard Sampling.''} \emph{PLOS Biology} 17 (6):
e3000357. \url{https://doi.org/10.1371/journal.pbio.3000357}.

\leavevmode\hypertarget{ref-Cazelles2016IntBio}{}%
Cazelles, Kévin, Nicolas Mouquet, David Mouillot, and Dominique Gravel.
2016. {``On the Integration of Biotic Interaction and Environmental
Constraints at the Biogeographical Scale.''} \emph{Ecography} 39 (10):
921--31. \url{https://doi.org/10.1111/ecog.01714}.

\leavevmode\hypertarget{ref-Dallas2020AbuNot}{}%
Dallas, Tad, Samuel Pironon, and Luca Santini. 2020. {``The
Abundant-Centre Is Not All That Abundant: A Comment to Osorio-Olvera Et
Al. 2020,''} 2020.02.27.968586.
\url{https://doi.org/10.1101/2020.02.27.968586}.

\leavevmode\hypertarget{ref-Daru2020GreToo}{}%
Daru, Barnabas H. 2020. {``GreenMaps: A Tool for Addressing the
Wallacean Shortfall in the Global Distribution of Plants.''}
\emph{bioRxiv}, 2020.02.21.960161.
\url{https://doi.org/10.1101/2020.02.21.960161}.

\leavevmode\hypertarget{ref-Dattilo2020SpeDri}{}%
Dáttilo, Wesley, Nathalia Barrozo-Chávez, Andrés Lira-Noriega, Roger
Guevara, Fabricio Villalobos, Diego Santiago-Alarcon, Frederico Siqueira
Neves, Thiago Izzo, and Sérvio Pontes Ribeiro. 2020. {``Species-Level
Drivers of Mammalian Ectoparasite Faunas.''} \emph{Journal of Animal
Ecology} 89 (8): 1754--65.
\url{https://doi.org/10.1111/1365-2656.13216}.

\leavevmode\hypertarget{ref-Dattilo2018EcoNet}{}%
Dáttilo, Wesley, and Victor Rico-Gray, eds. 2018. \emph{Ecological
Networks in the Tropics: An Integrative Overview of Species Interactions
from Some of the Most Species-Rich Habitats on Earth}. 1st ed. 2018.
Cham: Springer International Publishing : Imprint: Springer.
\url{https://doi.org/10.1007/978-3-319-68228-0}.

\leavevmode\hypertarget{ref-Dobson2009FooStr}{}%
Dobson, Andy. 2009. {``Food-Web Structure and Ecosystem Services:
Insights from the Serengeti.''} \emph{Philosophical Transactions of the
Royal Society B: Biological Sciences} 364 (1524): 1665--82.
\url{https://doi.org/10.1098/rstb.2008.0287}.

\leavevmode\hypertarget{ref-Draper1995AssPro}{}%
Draper, D. 1995. {``Assessment and Propagation of Model Uncertainty.''}
\emph{Journal of the Royal Statistical Society Series B-Statistical
Methodology} 57 (1): 45--97.
\url{https://doi.org/10.1111/j.2517-6161.1995.tb02015.x}.

\leavevmode\hypertarget{ref-Fourcade2016ComSpe}{}%
Fourcade, Yoan. 2016. {``Comparing Species Distributions Modelled from
Occurrence Data and from Expert-Based Range Maps. Implication for
Predicting Range Shifts with Climate Change.''} \emph{Ecological
Informatics} 36: 8--14.
\url{https://doi.org/10.1016/j.ecoinf.2016.09.002}.

\leavevmode\hypertarget{ref-Fricke2022EffDef}{}%
Fricke, Evan C., Alejandro Ordonez, Haldre S. Rogers, and Jens-Christian
Svenning. 2022. {``The Effects of Defaunation on Plants' Capacity to
Track Climate Change.''} \emph{Science}.
\url{https://doi.org/10.1126/science.abk3510}.

\leavevmode\hypertarget{ref-Godsoe2012HowSpe}{}%
Godsoe, William, and Luke J. Harmon. 2012. {``How Do Species
Interactions Affect Species Distribution Models?''} \emph{Ecography} 35
(9): 811--20. \url{https://doi.org/10.1111/j.1600-0587.2011.07103.x}.

\leavevmode\hypertarget{ref-Godsoe2017IntBio}{}%
Godsoe, William, Jill Jankowski, Robert D. Holt, and Dominique Gravel.
2017. {``Integrating Biogeography with Contemporary Niche Theory.''}
\emph{Trends in Ecology and Evolution} 32 (7): 488--99.
\url{https://doi.org/10.1016/j.tree.2017.03.008}.

\leavevmode\hypertarget{ref-Herkt2017MacCon}{}%
Herkt, K. Matthias B., Andrew K. Skidmore, and Jakob Fahr. 2017.
{``Macroecological Conclusions Based on IUCN Expert Maps: A Call for
Caution.''} \emph{Global Ecology and Biogeography} 26 (8): 930--41.
\url{https://doi.org/10.1111/geb.12601}.

\leavevmode\hypertarget{ref-Hortal2008UncMea}{}%
Hortal, Joaquín. 2008. {``Uncertainty and the Measurement of Terrestrial
Biodiversity Gradients.''} \emph{Journal of Biogeography} 35 (8):
1335--36. \url{https://doi.org/10.1111/j.1365-2699.2008.01955.x}.

\leavevmode\hypertarget{ref-Hortal2015SevSho}{}%
Hortal, Joaquín, Francesco de Bello, José Alexandre F. Diniz-Filho,
Thomas M. Lewinsohn, Jorge M. Lobo, and Richard J. Ladle. 2015. {``Seven
Shortfalls That Beset Large-Scale Knowledge of Biodiversity.''}
\emph{Annual Review of Ecology, Evolution, and Systematics} 46 (1):
523--49. \url{https://doi.org/10.1146/annurev-ecolsys-112414-054400}.

\leavevmode\hypertarget{ref-Hortal2008HisBia}{}%
Hortal, Joaquín, Alberto Jiménez-Valverde, José F. Gómez, Jorge M. Lobo,
and Andrés Baselga. 2008. {``Historical Bias in Biodiversity Inventories
Affects the Observed Environmental Niche of the Species.''} \emph{Oikos}
117 (6): 847--58.
\url{https://doi.org/10.1111/j.0030-1299.2008.16434.x}.

\leavevmode\hypertarget{ref-IUCNSSCRedListTechnicalWorkingGroup2021MapSta}{}%
IUCN SSC Red List Technical Working Group. 2021. \emph{Mapping Standards
and Data Quality for IUCN Red List Spatial Data}.

\leavevmode\hypertarget{ref-Krofel2021ResTax}{}%
Krofel, M., J. Hatlauf, W. Bogdanowicz, L. a. D. Campbell, R. Godinho,
Y. V. Jhala, A. C. Kitchener, et al. 2021. {``Towards Resolving
Taxonomic Uncertainties in Wolf, Dog and Jackal Lineages of Africa,
Eurasia and Australasia.''} \emph{Journal of Zoology} n/a (n/a): 1--14.
\url{https://doi.org/10.1111/jzo.12946}.

\leavevmode\hypertarget{ref-Ladle2013MapSpe}{}%
Ladle, Richard, and Joaquín Hortal. 2013. {``Mapping Species
Distributions: Living with Uncertainty.''} \emph{Frontiers of
Biogeography} 5 (1): 4--6.

\leavevmode\hypertarget{ref-McNaughton1992ProDis}{}%
McNaughton, S. J. 1992. {``The Propagation of Disturbance in Savannas
Through Food Webs.''} \emph{Journal of Vegetation Science} 3 (3):
301--14. \url{https://doi.org/10.2307/3235755}.

\leavevmode\hypertarget{ref-Meyer2016MulBia}{}%
Meyer, Carsten, Patrick Weigelt, and Holger Kreft. 2016.
{``Multidimensional Biases, Gaps and Uncertainties in Global Plant
Occurrence Information.''} \emph{Ecology Letters} 19 (8): 992--1006.
\url{https://doi.org/10.1111/ele.12624}.

\leavevmode\hypertarget{ref-Pocock2015BioRec}{}%
Pocock, Michael J. O., Helen E. Roy, Chris D. Preston, and David B. Roy.
2015. {``The Biological Records Centre: A Pioneer of Citizen Science.''}
\emph{Biological Journal of the Linnean Society} 115 (3): 475--93.
\url{https://doi.org/10.1111/bij.12548}.

\leavevmode\hypertarget{ref-Poisot2016ManMak}{}%
Poisot, Timothée, Benjamin Baiser, Jennifer A Dunne, Sonia Kéfi,
François Massol, Nicolas Mouquet, Tamara N Romanuk, Daniel B Stouffer,
Spencer A Wood, and Dominique Gravel. 2016. {``Mangal - Making
Ecological Network Analysis Simple.''} \emph{Ecography} 39 (4): 384--90.

\leavevmode\hypertarget{ref-Poisot2021GloKno}{}%
Poisot, Timothée, Gabriel Bergeron, Kevin Cazelles, Tad Dallas,
Dominique Gravel, Andrew MacDonald, Benjamin Mercier, Clément Violet,
and Steve Vissault. 2021. {``Global Knowledge Gaps in Species
Interaction Networks Data.''} \emph{Journal of Biogeography} 48 (7):
1552--63. \url{https://doi.org/10.1111/jbi.14127}.

\leavevmode\hypertarget{ref-Poisot2020EnvBia}{}%
Poisot, Timothée, Gabriel Bergeron, Kevin Cazelles, Tad Dallas,
Dominique Gravel, Andrew Macdonald, Benjamin Mercier, Clément Violet,
and Steve Vissault. 2020. {``Environmental Biases in the Study of
Ecological Networks at the Planetary Scale.''} \emph{bioRxiv},
2020.01.27.921429. \url{https://doi.org/10.1101/2020.01.27.921429}.

\leavevmode\hypertarget{ref-Power1992TopBot}{}%
Power, Mary E. 1992. {``Top-Down and Bottom-Up Forces in Food Webs: Do
Plants Have Primacy.''} \emph{Ecology} 73 (3): 733--46.
\url{https://doi.org/10.2307/1940153}.

\leavevmode\hypertarget{ref-Pringle2016LarHer}{}%
Pringle, Robert M., Kirsten M. Prior, Todd M. Palmer, Truman P. Young,
and Jacob R. Goheen. 2016. {``Large Herbivores Promote Habitat
Specialization and Beta Diversity of African Savanna Trees.''}
\emph{Ecology} 97 (10): 2640--57.
\url{https://doi.org/10.1002/ecy.1522}.

\leavevmode\hypertarget{ref-Ronquillo2020AssSpa}{}%
Ronquillo, Cristina, Fernanda Alves-Martins, Vicente Mazimpaka, Thadeu
Sobral-Souza, Bruno Vilela-Silva, Nagore G. Medina, and Joaquín Hortal.
2020. {``Assessing Spatial and Temporal Biases and Gaps in the Publicly
Available Distributional Information of Iberian Mosses.''}
\emph{Biodiversity Data Journal} 8: e53474.
\url{https://doi.org/10.3897/BDJ.8.e53474}.

\leavevmode\hypertarget{ref-Roy2016FocPla}{}%
Roy, Helen E., Elizabeth Baxter, Aoine Saunders, and Michael J. O.
Pocock. 2016. {``Focal Plant Observations as a Standardised Method for
Pollinator Monitoring: Opportunities and Limitations for Mass
Participation Citizen Science.''} \emph{PLOS ONE} 11 (3): e0150794.
\url{https://doi.org/10.1371/journal.pone.0150794}.

\leavevmode\hypertarget{ref-Ryan2018RolCit}{}%
Ryan, S. F., N. L. Adamson, A. Aktipis, L. K. Andersen, R. Austin, L.
Barnes, M. R. Beasley, et al. 2018. {``The Role of Citizen Science in
Addressing Grand Challenges in Food and Agriculture Research.''}
\emph{Proceedings of the Royal Society B: Biological Sciences} 285
(1891). \url{https://doi.org/10.1098/rspb.2018.1977}.

\leavevmode\hypertarget{ref-Scott2018RolHer}{}%
Scott, Abigail L., Paul H. York, Clare Duncan, Peter I. Macreadie, Rod
M. Connolly, Megan T. Ellis, Jessie C. Jarvis, Kristin I. Jinks, Helene
Marsh, and Michael A. Rasheed. 2018. {``The Role of Herbivory in
Structuring Tropical Seagrass Ecosystem Service Delivery.''}
\emph{Frontiers in Plant Science} 9: 127.
\url{https://doi.org/10.3389/fpls.2018.00127}.

\leavevmode\hypertarget{ref-Stoyanov2020CraVar}{}%
Stoyanov, S. 2020. {``Cranial Variability and Differentiation Among
Golden Jackals (Canis Aureus) in Europe, Asia Minor and Africa.''}
\emph{ZooKeys}. \url{https://doi.org/10.3897/zookeys.917.39449}.

\leavevmode\hypertarget{ref-Strydom2021RoaPre}{}%
Strydom, Tanya, Michael D. Catchen, Francis Banville, Dominique Caron,
Gabriel Dansereau, Philippe Desjardins-Proulx, Norma R. Forero-Muñoz, et
al. 2021. {``A Roadmap Towards Predicting Species Interaction Networks
(across Space and Time).''} \emph{Philosophical Transactions of the
Royal Society B: Biological Sciences} 376 (1837): 20210063.
\url{https://doi.org/10.1098/rstb.2021.0063}.

\leavevmode\hypertarget{ref-Svenning2014InfInt}{}%
Svenning, Jens Christian, Dominique Gravel, Robert D. Holt, Frank M.
Schurr, Wilfried Thuiller, Tamara Münkemüller, Katja H. Schiffers, et
al. 2014. {``The Influence of Interspecific Interactions on Species
Range Expansion Rates.''} \emph{Ecography} 37 (12): 1198--1209.
\url{https://doi.org/10.1111/j.1600-0587.2013.00574.x}.

\leavevmode\hypertarget{ref-Tsunoda2020VarTro}{}%
Tsunoda, Hiroshi, and Masayuki U. Saito. 2020. {``Variations in the
Trophic Niches of the Golden Jackal Canis Aureus Across the Eurasian
Continent Associated with Biogeographic and Anthropogenic Factors.''}
\emph{Journal of Vertebrate Biology} 69 (4): 20056.1.
\url{https://doi.org/10.25225/jvb.20056}.

\leavevmode\hypertarget{ref-Valiente-Banuet2015SpeLos}{}%
Valiente-Banuet, Alfonso, Marcelo A. Aizen, Julio M. Alcántara, Juan
Arroyo, Andrea Cocucci, Mauro Galetti, María B. García, et al. 2015.
{``Beyond Species Loss: The Extinction of Ecological Interactions in a
Changing World.''} Edited by Marc Johnson. \emph{Functional Ecology} 29
(3): 299--307. \url{https://doi.org/10.1111/1365-2435.12356}.

\leavevmode\hypertarget{ref-Young2013EffMam}{}%
Young, Hillary S., Douglas J. McCauley, Kristofer M. Helgen, Jacob R.
Goheen, Erik Otárola-Castillo, Todd M. Palmer, Robert M. Pringle, Truman
P. Young, and Rodolfo Dirzo. 2013. {``Effects of Mammalian Herbivore
Declines on Plant Communities: Observations and Experiments in an
\textsc{A} Frican Savanna.''} Edited by Luis Santamaria. \emph{Journal
of Ecology} 101 (4): 1030--41.
\url{https://doi.org/10.1111/1365-2745.12096}.

\end{CSLReferences}

\end{document}
